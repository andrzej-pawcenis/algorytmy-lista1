\documentclass{article}
\usepackage[utf8]{inputenc} 
\usepackage[T1]{fontenc}
\usepackage{listings}
\usepackage{amsmath}
\usepackage{amsfonts}
\usepackage{algpseudocode}
\usepackage{lmodern}  
\title{Algorytmy lista 1}
\begin{document}
	\maketitle
	\section{Wstęp}
		Na potrzeby zadania 1 zamieszczono podane w prezentacji algorytmy MAX oraz połówkowy algorytm MAX
			\subsection{Algorytm MAX}
				\begin{algorithmic}[1]
					\Require n>=1
					\Require \(x_1\) \dots \(x_n\)
					\State Wynik y - największa spośród liczb \(x_1\) \dots \(x_n\)
					\State[Krok 1] y:=\(x_1\)
					\State[Krok 2] i:=1
					\If{i>n}
						\State[Krok 3] Idź do krok 7
					\EndIf
					\If{(\(x_i\) > y}
						\State[Krok 4]] y:=\(x_i\)
					\EndIf	
					\State[Krok 5] i:=i+1
					\State[Krok 6] Przejdź do kroku 3
					\State[Krok 7] Szukana liczbą jest y (KONIEC)
				\end{algorithmic}[1]
			\subsection{Połówkowy Algorytm MAX}
				\begin{algorithmic}[2]
				 	\Require $n \in \mathbb{N}$ i n liczb \(a_1\), \(a_2\),\dots,\(a_n\) 
				 	\State Wynik max - największa spośród liczb
				 	\(a_1\), \(a_2\),\dots,\(a_n\) 
				 	\State M:=$\emptyset$ (M: zbiór kandydatów na maximum, na początku pusty)
				 	\If{n jest parzysta}
				 		\State k:=n-1
				 	\Else
				 		\State k:=n-2
				 	\EndIf
				 	\For{$n = 1,2, \dots, k$}
						\If{\(a_i\) <= \(a_{i+1}\)}
							\State M:= $M \cup \{a_{i+1}\}$
						\Else
							\State M := $M \cup \{a_i\}$	
						\EndIf
				 	\EndFor
				 	\If{n jest liczbą parzystą}
				 		\State M := $M \cup \{a_i\}$
				 	\EndIf
				 	\State Znajdź maksimum max w zbiorze M, stosując prosty algorytm MAX
				 	\State Koniec, max zawiera poszukiwaną wartość
				\end{algorithmic}[2]
		\section{Zadanie 1}
			\subsection{pkt 1}
				\begin{algorithmic}[3]
					\Require n>=1
					\Require \(x_1\) \dots \(x_n\)
					\State Wynik y - największa spośród liczb \(x_1\) \dots \(x_n\)
					\State[Krok 1] y:=\(x_1\)
					\State[Krok 2] i:=1
					\For{$i = 1,2, \dots, n$}
					\If{(\(x_i\) > y}
					\State[Krok 3]] y:=\(x_i\)
					\EndIf	
					\EndFor
					\State[Krok 4] Szukana liczbą jest y (KONIEC)
				\end{algorithmic}[3]
			\subsection{pkt 2}
				\begin{algorithmic}[4]
					\Require $n \in \mathbb{N}$ i n liczb \(a_1\), \(a_2\),\dots,\(a_n\) 
					\State Wynik max - największa spośród liczb
					\(a_1\), \(a_2\),\dots,\(a_n\) 
					\State M:=$\emptyset$ (M: zbiór kandydatów na maximum, na początku pusty)
					\If{n jest parzysta}
					\State k:=n-1
					\Else
					\State k:=n-2
					\EndIf
					\If{n<=k}[pętla]
					\If{\(a_i\) <= \(a_{i+1}\)}
					\State M:= $M \cup \{a_{i+1}\}$
					\Else
					\State M := $M \cup \{a_i\}$	
					\EndIf
					\EndIf
					\State n:=n+1
					\If{n==k}
					\State Idź do [pętla]
					\EndIf
					\If{n jest liczbą parzystą}
					\State M := $M \cup \{a_i\}$
					\EndIf
					\State Znajdź maksimum max w zbiorze M, stosując prosty algorytm MAX
					\State Koniec, max zawiera poszukiwaną wartość
				\end{algorithmic}[4]
		\section{Zadanie 2}
			\subsection{Cel}
				Celem algorytmu jest porównanie dwóch ciągów znaków zawierających cyfry.
			\subsection{Wejście}
				Algorytm przyjmuje dwa ciągi znaków (numeric).
			\subsection{Wyjście}
				Zmienna typu zawierająca wartość większej liczby
			\subsection{Założenia}
				Oba ciągi znaków zawierają tylko cyfry
			\subsection{Kroki działania}
				\begin{algorithmic}[5]
					\Require $str1 \in [0-9]$
					\Require $str2 \in [0-9]$
					\State lStr1:=len(str1)
					\State lStr2:=len(str2)
					\State greaterNumber:=str1
					\If{lStr1 > lStr2}
					\State greaterNumber:=lStr1
					\ElsIf{lStr1 < lStr2}
					\State greaterNumber:=lStr2
					\ElsIf{lStr1 == lStr2}
					\For{$i \in 0,1,\dots, lStr1$}
					\If{str1[i] == str2[i]}
					\State continue
					\ElsIf{str1[i] > str2[i]}
					\State greaterNumber:=lStr1
					\State break
					\ElsIf{str1[i] < str2[i]}
					\State greaterNumber:=lStr2
					\State break
					\EndIf
					\EndFor
					\EndIf
					\State Koniec, zmienna greaterNumber zawiera wartość większej liczby
				\end{algorithmic}[5]
		\section{Zadanie 3}
			\subsection{pseudokod}
				\begin{algorithmic}[6]
					\Require $str1 \in [a-z]$
					\Require $str2 \in [a-z]$
					\State order:=[]
					\State lStr1:=len(str1)
					\State lStr2:=len(str2)
					\If{lStr1 == lStr2}
						\For{$i \in 1,\dots,lStr1$}
							\If{ord(str1[i]) == ord(str2[i])}
								\State continue
							\ElsIf{ord(str1[i] > ord(str2[i]))}
								\State order[0]:=str2
								\State order[1]:=str1
								\State break
							\ElsIf{ord(str1[i] < ord(str2[i]))]}
								\State order[0]:=str1
								\State order[1]:=str2
								\State break
							\EndIf
						\EndFor
					\ElsIf{lStr1 < lStr2}
						\For{$i \in 1,\dots,lStr1$}
							\If{ord(str1[i]) == ord(str2[i])}
								\If{i==lStr1}
									\State order[0]:=str1
									\State order[1]:=str2
									\State break
								\Else
									\State continue
								\EndIf
							\ElsIf{ord(str1[i] > ord(str2[i]))}
								\State order[0]:=str2
								\State order[1]:=str1
								\State break
							\ElsIf{ord(str1[i] < ord(str2[i]))]}
								\State order[0]:str1
								\State order[1]:=str2
								\State break
							\EndIf
						\EndFor
					\ElsIf{lStr1 > lStr2}
						\For{$i \in 1,\dots,lStr2$}
							\If{ord(str1[i]) == ord(str2[i])}
								\If{i==lStr2}
			 						\State order[0]:=str2
			 						\State order[0]:=str1
									\State break
								\Else
									\State continue
								\EndIf
							\ElsIf{ord(str1[i] > ord(str2[i]))}
								\State order[0]:=str2
								\State order[1]:=str1
								\State break
							\ElsIf{ord(str1[i] < ord(str2[i]))]}
								\State order[0]:str1
								\State order[1]:=str2
								\State break
							\EndIf
						\EndFor
					\EndIf
					\State Koniec, zmienna tablicowa order zawiera kolejność słów
				\end{algorithmic}[6]
			\section{Złożoność pamięciowa}
				Nie wykorzystujemy w tym algorytmie pomocnicznych tablic, jedyne dodatkowe zmienne to długości słów oraz zmienna przechowująca porządek słów. Więc złożoność pamięciowa jest w przybliżeniu liniowa O(n)
				Złożoność czasowa zależy tylko od długości dłuższego słowa więc złożoność czasowa wynosi również O(n)
		\section{Zadanie 4}
			\subsection{Cel}
			Celem algorytmu jest znalezienie liczb pierwszych od podane liczby
			\subsection{Wejście}
			Algorytm przyjmuje liczbę.
			\subsection{Wyjście}
			Zmienna tablicowa której indeksy są liczbami a wartości informacją czy liczba jest pierwsza czy nie
			\subsection{Założenia}
			Podana liczba jest większa od 1
			\subsection{Kroki}
			\begin{algorithmic}[5]
				\Require $n \in  \mathbb{N}$
				\Ensure $n > 1$
				\State A tablica wartości logicznych o wielkości n
				\For{i:=2,3,\dots,$\sqrt{n}$}
					\If{A[i] = true}
						\For{j:=2*i,3*i,\dots,n}
							\State A[j] := false
						\EndFor
					\EndIf
				\EndFor
				\State Wyjście: indeksy tablicy A zawierające wartość true
			\end{algorithmic}[5]
		\section{Złożoność}
			Złożoność pamięciowa O(n) ponieważ wielkość wynikowej tablicy zależy liniowo.
			Złożoność czasowa wynosi O(n*log(log(n)))
		\section{Implementacja}
			\begin{lstlisting}[language=Python, caption=implementacja sita Erostenesa]
				import math
				number=100
				
				sieve = []
				sieve = [True for i in range(number)]
				
				sieve[0] = False
				sieve[1] = False
				
				for i in range(2, int(math.ceil(math.sqrt(number)))):
				if sieve[i]:
				k = 2
				j = k * i
				while j < number:
				sieve[j] = False
				k = k + 1
				j = i * k
				
				primes = [idx for idx, element in enumerate(sieve) if element == True]
				print(primes)
			\end{lstlisting}
\end{document}
