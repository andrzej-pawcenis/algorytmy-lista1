\documentclass{article}
\usepackage[utf8]{inputenc} 
\usepackage[T1]{fontenc}
\usepackage{listings}
\usepackage{amsmath}
\usepackage{amsfonts}
\usepackage{algpseudocode}
\usepackage{lmodern}  
\usepackage{graphicx}
\title{Algorytmy lista 2}
\author{Andrzej Pawcenis}
\begin{document}
	\maketitle
	\section{Zadanie 1}
		\subsection{Algorytm is\_sorted}
			\begin{algorithmic}[1]
				\State pomocniczy algorytm do sprawdzania czy tablica jest posortowana rosnąco, roboczo nazwiemy go is\_sorted
				\Require t = [\(x_1\) \dots \(x_n\)]
				\State bool flag znacznik posortowania
				\State bool =: True
				\For{$n = 1, 2 \cdots n-1$}
					\If{t[n] > t[n+1]}
						\State flag := False
						\State GOTO[RES]
					\EndIf
				\EndFor
				\State[RES] Koniec zmiana flag zawiera wartość typu bool informującą o tym czy tablica jest posortowana
			\end{algorithmic}[1]
		\subsection{bogosort}
			\begin{algorithmic}[2]
				\Require nieposortowana t = [\(x_1\) \dots \(x_n\)]
				\While{$is\_sorted(t)$}
					\State t = random(t)
				\EndWhile
				\State Koniec tablica t jest posortowana rosnąco
			\end{algorithmic}[2]
	\section{Zadanie 2}
		\subsection{Sortowanie przez wybór}
			\begin{lstlisting}[language=Python, caption=implemtacja algorytmu sortowania przez wybór]
def s_sort(arr):
	for i in range(0, len(arr)):
	min_index = i
	for j in range(i+1,len(arr)):
		if arr[j] < arr[min_index]:
			min_index = j
	buff = arr[i]
	arr[i] = arr[min_index]
	arr[min_index] = buff
	print(arr)

arr = [5,6,0,8,8,3]
s_sort(arr)
			\end{lstlisting}
		\subsection{kroki:}
		\begin{lstlisting}
Dane wejsciowe: [5,6,0,8,8,3]
Kroki:
		[0, 6, 5, 8, 8, 3]
		[0, 3, 5, 8, 8, 6]
		[0, 3, 5, 8, 8, 6]
		[0, 3, 5, 6, 8, 8]
		[0, 3, 5, 6, 8, 8]
		[0, 3, 5, 6, 8, 8]
\end{lstlisting}
	\section{Zadanie 3}
		\subsection{quick sort}
			\begin{lstlisting}[language=python, caption={implementacja quick sort}]
def quick_sort(arr, _min, _max):
	if _min < _max:
		p_index = partition(arr, _min, _max)
			quick_sort(arr, _min, p_index - 1)
			print(f"lewy fragment: {arr}")
			quick_sort(arr, p_index + 1, _max)
			print(f"prawy fragment: {arr}")

def partition(arr, _min, _max):
	pivot = arr[_max]
	i = _min - 1
	for j in range(_min, _max - 1):
		if arr[j] < pivot:
			i = i + 1
			buff = arr[i]
			arr[i] = arr[j]
			arr[j] = buff
	buff = arr[i + 1]
	arr[i + 1] = arr[_max]
	arr[_max] = buff
	
	return i + 1

arr = [5,6,0,8,3]
print(f"dane wejsciowe: {arr}")
quick_sort(arr, 0, len(arr)-1)
print(f"wynik: {arr}")
			\end{lstlisting}
		\subsection{Kroki:}
			\begin{lstlisting}
dane wejsciowe: [5, 6, 0, 8, 3]
lewy fragment: [0, 3, 5, 8, 6]
lewy fragment: [0, 3, 5, 6, 8]
prawy fragment: [0, 3, 5, 6, 8]
prawy fragment: [0, 3, 5, 6, 8]
wynik: [0, 3, 5, 6, 8]
			\end{lstlisting}
	\section{Zadanie 4}
		\subsection{Scalanie tablic}
			\begin{lstlisting}[language=python, caption={implementacja scalania tablic}]
def merge(arr, left, right):
	i = 0
	j = 0
	k = 0
	while i < len(left) and j < len(right):
		if left[i] > right[j]:
			arr[k] = left[i]
			i = i + 1
			print(arr)
		else:
			arr[k] = right[j]
			j = j + 1
			print(arr)
		k = k + 1

	while i < len(left):
		arr[k] = left[i]
		i = i + 1
		k = k + 1
		print(arr)
	while j < len(right):
		arr.append(right[j])
		j = j + 1
		k = k + 1
		print(arr)
		
arr = [8, 6, 5, 3, 0, 9, 5, 3]
arr1 = [8, 6, 5, 3, 0]
arr2 = [9, 5, 3]

merge(arr, arr1, arr2)
			\end{lstlisting}
		\subsection{Kroki}
			\begin{lstlisting}
[9, 6, 5, 3, 0, 9, 5, 3]
[9, 8, 5, 3, 0, 9, 5, 3]
[9, 8, 6, 3, 0, 9, 5, 3]
[9, 8, 6, 5, 0, 9, 5, 3]
[9, 8, 6, 5, 5, 9, 5, 3]
[9, 8, 6, 5, 5, 3, 5, 3]
[9, 8, 6, 5, 5, 3, 3, 3]
[9, 8, 6, 5, 5, 3, 3, 0]
			\end{lstlisting}
	\section{Zadanie 5}
		\subsection{Zliczanie}
			\begin{lstlisting}[language=python, caption={implementacja tablicy zliczajacej}]
def counting_table(array, max_value):
	c_table = [0 * i for i in range(0, max_value +1)]
	for i in range(0, len(array)):
		c_table[arr[i]] += 1

	return c_table

arr = [5,6,0,8,3]
print(counting_table(arr, 8))
			\end{lstlisting}
		\subsection{wynik}
			[1, 0, 0, 1, 0, 1, 1, 0, 1]
	\section{Zadanie 6}
		\subsection{Graf miast}
			\begin{figure}[h]
				\centering
				\includegraphics[width=0.7\textwidth]{./cities.png}
				\caption{Graf miast}
				\label{fig:obraz}
			\end{figure}
	
	\section{Zadanie 6.1}
		Trasę Wrocław - Koło można wytyczyć tak, że droga będzie krótsza jednak czas dłuższy lub też wydłużyć drogę skracając jednocześnie czas.
		\begin{figure}
			\centering
			\includegraphics[width=0.7\linewidth]{screenshot001}
			\caption{}
			\label{fig:screenshot001}
		\end{figure}
	\section{Podsumowanie}
		Źródła tex oraz python dla tego dokumentu można znaleźć pod adresem https://github.com/andrzej-pawcenis/algorytmy-lista1		
\end{document}